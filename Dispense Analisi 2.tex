\documentclass[a4paper; 11pt oneside]{book}
\usepackage[italian]{babel}
\usepackage[margin = 1.5 in]{geometry}

\usepackage{tikz}

\usepackage{amsmath}
\usepackage{amssymb}

\usepackage{amsthm}
\newtheoremstyle{break}% name
	{5pt}%         Space above, empty = `usual value'
	{5pt}%         Space below
	{\itshape}% Body font
	{10pt}%         Indent amount (empty = no indent, \parindent = para indent)
	{\bfseries}% Thm head font
	{}%        Punctuation after thm head
	{\newline}% Space after thm head
	{}%         Thm head spec

\theoremstyle{break}
\newtheorem{teo}{Teorema}[chapter]
\newtheorem{cor}{Corollario}[teo]
\newtheorem{lemma}[teo]{Lemma}
\newtheorem{defin}[teo]{Definizione}
\newtheoremstyle{peresempio}% name
	{5pt}%         Space above, empty = `usual value'
	{5pt}%         Space below
	{}% Body font
	{}%         Indent amount (empty = no indent, \parindent = para indent)
	{\bfseries}% Thm head font
	{}%        Punctuation after thm head
	{\newline}% Space after thm head
	{}%         Thm head spec
\theoremstyle{peresempio}
\newtheorem{es}[cor]{Esempio}
\newtheorem{oss}[cor]{Osservazione}

\setlength\parindent{0pt}

\everymath{\displaystyle}

\usepackage{hyperref}
\hypersetup{
    colorlinks=true,
    linkcolor=blue,
    filecolor=magenta,      
    urlcolor=cyan,
    }
\urlstyle{same}


\title{Analisi 2}
\author{Emilio Groppi}

\begin{document}
\maketitle
\tableofcontents
%\pagebreak
\chapter{Integrali generalizzati o Impropri}
\section{Funzioni localmente integrabili}

Nel definire l'integrale di Riemann $\int_a^b f(x)dx$ si è supposto che $f$ fosse una funzione limitata definita su un intervallo chiuso e limitato (cioè compatto). Vogliamo ora rimuovere queste rstizioni.
\begin{defin}
Una funzione $f$ si dice \textbf{localmente integrabile} sull'intervallo $J$ qualunque se $f$ è integrabile $\forall K_{compatto}\subseteq J$
\end{defin}

\begin{es}
Se $f:J\to\mathbb{R}$ continua, allora $f$ è localmente integrabile su $J$.
\end{es}
Osservazione:
Sia $f:J\to\mathbb{R}$ localmente integrabile e sia $c\in J$ finita. Allora la funzione integrale
$$F(x)=\int_c^x f(t)dt \,\, con \;x\in J $$
è continua in $J$.

Per ogni $d\in J$ si ha che $\lim{x\to d}\int_c^x f(t)dt=\lim_{x\to d} F(x)=F(d)=\int_c^d f(t)dt$
\section{Funzioni integrabili in senso generalizzato}
\textbf{Idea}\newline
Se $d$ è un punti do accolmulazione per $J$, una $d\notin J$, si usa u linguaggio al limite.

Distinguiamo i vari casi:
    \begin{enumerate}
      \item Sia $J=[a,b[$ con $b\in \mathbb{R}\cup\{+\inf\}$ e sia $f:J\to\mathbb{R}$ localmente integrabile su $J$. Si dice che $f$ è intregabile in senso generalizzato su $J$ se esiste finito il limite $\displaystyle \lim_{x\to b}\int_a^x f(x)dt:=\int_a^b f(t)dt$

      \item Sia $J=]a,b[$ con $a\in \mathbb{R}\cup\{-\inf\}$ e sia $f:J\to\mathbb{R}$ localmente integrabile su $J$. Si dice che $f$ è intregabile in senso generalizzato su $J$ se esiste finito il limite $\displaystyle \lim_{x\to a}\int_x^b f(x)dt:=\int_a^b f(t)dt$ 
      
      \item Sia $J=]a,b[$ con $a\in \mathbb{R}\cup\{-\inf\}$ e $b\in \mathbb{R}\cup\{+\inf\}$ e sia $f$ una funzione localmente integrabile su $J$. Si dice che $f$ è integrabile in senso generalizzato su $J$ se esiste $c\in J$ tale che $f$ è integravile in senso generalizzato su $]a,c]$ e $[c,b[$ e si pone
      $$\int_a^b f(t)dt:=\int_a^c f(t)dt + \int_c^b f(t)dt$$
    \end{enumerate}
  \begin{oss}
  La definizione 3 non dipende da $c$.
  \end{oss}
  \section{Integrabilità in senso generalizzato delle funzioni campione} %pag 4
  \begin{teo}[$J$ illimitato]
  \begin{enumerate}
    \item Sia $J=[a,+\infty[,$ con $ a>0$ si ha che:
    
    $$ 
    \int_a^{+ \infty}\frac{1}{x^{\alpha}}dx \text{ esiste finito }\Leftrightarrow\alpha>1 
$$
    \item Sia $J=]- \infty, b]$ con $b<0$, si ha che:
    $$
    	\int_{- \infty}^b\frac{1}{x^{\alpha}}dx \text{ esiste finito }\Leftrightarrow\alpha>1 
    $$
  \end{enumerate}
  
  \end{teo}
  
  \begin{proof}
  \begin{enumerate}
  \item si ha
  $$\int_a^{+ \infty}\frac{1}{x^{\alpha}}dx=\begin{cases}
  \left[\frac{1}{1-\alpha}t^{1-\alpha}\right]^x_a=\frac{1}{1-\alpha}\left(x^{1-\alpha}-a^{1-\alpha}\right)\;\;se\;\alpha\neq 1 \\ \\
  \left[\log t\right]^x_a=\log x-\log a\;\; se\; \alpha=1
  \end{cases}
  $$
  E quindi il limiti per $x\to + \infty$ è finito sse $\alpha>1$
  \item Simile
  \end{enumerate}
  \end{proof}

  \begin{teo}[$J$ limitato]
  \begin{enumerate}
    \item Sia $J=[a,b[,$ con $ b\in\mathbb{R}$ si ha che:
    
    $$ 
    \int_a^{b}\frac{1}{(b-x)^{\alpha}}dx \text{ esiste finito }\Leftrightarrow\alpha>1 
$$
    \item Sia $J=]a, b]$ con $a\in\mathbb{R}$, si ha che:
    $$
    	\int_{a}^b\frac{1}{(x-a)^{\alpha}}dx \text{ esiste finito }\Leftrightarrow\alpha<1 
    $$
  \end{enumerate}
  
  \end{teo}
  
  \begin{proof}
  \begin{enumerate}
  \item si ha
  $$\int_a^{x}\frac{1}{x^{\alpha}}dx=
  \begin{cases}
  \left[\frac{-1}{1-\alpha}(b-t)^{1-\alpha}\right]^x_a=\frac{1}{1-\alpha} \left((b-a)^{1-\alpha}-(b-x)^{1-\alpha}\right)\;\;se\;\alpha\neq 1 \\ \\
  \left[\log (b-t)\right]^x_a=\log (b-a)-\log (b-x)\;\; se \;\alpha=1
  \end{cases}
  $$
  E quindi il limiti per $x\to + \infty$ è finito sse $\alpha>1$
  \item Simile
  \end{enumerate}
  \end{proof}
\section{Criteri di integrabilità}%5
\textbf{Premessa}\newline
Sappiamo ce esistono finiti $\int_0^1\frac{1}{x^\alpha}dx$ se $\alpha<1$ e $\int_1^{+\infty}\frac{1}{x^\alpha}dx$ se $\alpha>1$, ma er esempio non siamo ancora in grado di decidere se è finito $\int _{-\infty}^b e^{-x^2}dx$, in quanto la funzione $e^{-x^2}$ non ammette primitive elementari e quindi non possiamo calcolarlo esplicitamente.
\begin{oss}
$\Phi(x)=\frac{1}{\sqrt{2\pi}}\int_{-\infty}^xe^{-\frac{1}{2}t^2}dt$: Funzione di ripartizione della normale standard.
\end{oss}
Cerchiamo dei metodi che ci permettano di stabilire l'integrabilità in senso generalizzato senza dover calcolare direttamente il limite.
    \subsection{Criterio generale di Cauchy}
  \begin{teo}
  Sia $f:J=[a,b[\to \mathbb{R}$, con $b\in \mathbb{R}\cup\{+\infty\}$ localmente integrabile.\newline
    Si ha che $f$ è integrabile in senso generalizzato su $J\Leftrightarrow$ $$\forall\epsilon>0\exists I\text{ intotno di b} \; t.c. \;\forall x_1,x_2\in I\;se\; x_1,x_2\in I\Rightarrow\left|\int_{x_1}^{x_2}f(t)dt \right|<\epsilon$$
    \end{teo}
    \begin{proof}
    Posto $F(x)=\int_a^xf(t)dt$, esiste finito $\lim_{x\to b^-}F(x)$ se e solo se vale la condizione di Cauchy: $$\forall\epsilon>0\exists I\text{ intotno di b} \; t.c. \;\forall x_1,x_2\in I\;se\; x_1,x_2\in I\Rightarrow\left|F(x_2)-F(x_1) \right|<\epsilon$$
    dove $F(x_2)-F(x_1)=\int_{x_1}^{x_2}f(t)dt$
    \end{proof}
    Vale un analogo risultato se $J=]a,b]$ con $a\in\mathbb{R}\cup\{-\infty\}$
    \subsection{Aut-aut per l'integrale generalizzato}
    \begin{teo}
    Sia $f:J=[a,b[\to \mathbb{R}$, con $b\in \mathbb{R}\cup\{+\infty\}$ localmente integrabile e $f(x)\leq 0$ in $J$, allora esiste $$
    \lim_{x\to b}\int_a^x f(t)dt=\sup_{x\in J}\int_a^x f(t)dt
    $$
    \end{teo}
    \begin{proof}
    Poiché $f(x)\geq 0$ in $[a,b[$, si ha che per ogni $x_1,x_2\in[a,b[$, con $x_1<x_2$
    $$
    \int_a^{x_2} f(t)dt-\int_a^{x_1} f(t)dt=\int_{x_1}^{x_2} f(t)dt\geq 0
    $$
    e quindi $F(x)=\int_a^{x} f(t)dt$ è crescente (in senso debole). Dunque per il teorema del limite delle funzioni monotone, esiste
    $$
    \lim_{x\to b}F(x)=\sup_{x\in J}F(x)
    $$
    \end{proof}
    Vale un analogo risultato se $J=]a,b]$ con $a\in\mathbb{R}\cup\{-\infty\}$
    \begin{oss}
    Si può notare che in generale \textsc{non} esiste $\lim_{x\to b}\int_a^x f(t)dt$.\newline Per esempio, se $f(t)=\cos t$, non esiste $\lim_{x\to +\infty}\int_0^x \cos t dt=\lim_{x\to +\infty} \sin x$
    \end{oss}
    \subsection{Criterio del confronto}
    \begin{teo}
    Siano $f,g: J=[a,b[\to\mathbb{R}$ localmente integrabili e tali che $0\leq f(x)\leq g(x)$ in $J$. Si ha che:
    \begin{enumerate}
    \item Se $g$ è integrabile in senso generalizzato su $J$, allora lo è anche $f$ e $\int_a^b f(x)dx\leq\int_a^b g(x)dx$
    \item Se $f$ \textsc{non} è integrabile in senso generalizzato su $J$, allora non lo è neanche $g$.
    \end{enumerate}

    \end{teo}
    \begin{proof}
    \begin{itemize}
    \item Per ogni $x\in J$ si ha che:
    $$
    F(x)=\int_a^x f(t)dt\leq\int_a^x g(t)dt=G(x)
    $$
    Per il teorema dell'aut-aut, esiste
    $$
    \lim_{x\to b}F(x)=\sup_{x\in J} F(x)\leq\sup_{x\in J} G(x)=\lim_{x\to b}G(x)<+\infty
    $$
    e quindi $f$ è integrabile in senso generalizzato e 
    $$
    \int_a^b f(t)dt\leq\int_a^b g(t)dt
    $$
    \item Si tratta dell'implicazione contronominale della precedente
    \end{itemize}
    \end{proof}
    Vale un analogo risultato se $J=]a,b]$ con $a\in\mathbb{R}\cup\{-\infty\}$
    \begin{cor}[Criterio del confronto asintotico]
    Siano $f,g: J=[a,b[\to\mathbb{R}$ localmente integrabili e tali che $f(x)>0$ e $g(x)>0$ in $J$, ed esiste
    $$
    \lim_{x\to b} \frac{f(x)}{g(x)}=L\in ]0,+\infty[
    $$
    allora $f e g$ sono entramb integravili in senso generalizzato oppure nessuna delle due lo è
    \end{cor}
    \begin{proof}
    Dalla definizione di limite di deduce che esiste $c\in J$ tale che:
    $$
    \frac{1}{2}Lg(x)\leq f(x)\leq Lg(x)\;\;\;\;\forall x\in [c,b[
    $$
    Dal criterio del confronto segue la tesi.
    \end{proof}
    Vale un analogo risultato se $J=]a,b]$ con $a\in\mathbb{R}\cup\{-\infty\}$
  
  \section{Funzioni assolutamente e semplicemente integrabili in senso generalizzato}
    \end{document}